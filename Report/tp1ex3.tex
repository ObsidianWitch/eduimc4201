\subsection{Ex3 - Scheduling}

\subsubsection{Schedulability}

To study the schedulability of a task set, the processor load is computed. This value allows us to determine with which algorithm the task set is schedulable. \\

\noindent For the first task set, the processor load is the following : \\

$U_1 = \sum_{i=1}^{n} \frac{C_i}{T_i} = 0.958$ \\

The value of $U_1$ is greater than $n(2^{\frac{1}{n}}-1)$ (with $n$ the number of tasks) which is equal to 0.779. It is then not schedulable with RM, but $U_1$ is still lesser than 1, making the task set schedulable with EDF. \\

When it comes to the second task set, the processor load's value is $U_2 = 0.852$. Like the first task set, it is schedulable with EDF, but not with RM. \\

\begin{center}
	\begin{tabular}{|c|c|c|c|}
		\hline
		\multicolumn{4}{|c|}{\textbf{Set 1}} \\
		\hline
		\, & C & Period & Deadline \\
		\hline
		T1 & 1 & 4 & 4 \\
		\hline
		T2 & 2 & 6 & 6 \\
		\hline
		T3 & 3 & 8 & 8 \\
		\hline
	\end{tabular}
	\quad
	\begin{tabular}{|c|c|c|c|}
		\hline
		\multicolumn{4}{|c|}{\textbf{Set 2}} \\
		\hline
		\, & C & Period & Deadline \\
		\hline
		T1 & 2 & 7 & 7 \\
		\hline
		T2 & 2 & 11 & 11 \\
		\hline
		T3 & 5 & 13 & 13 \\
		\hline
	\end{tabular}
\end{center}

\subsubsection{Rate-monotonic scheduling}

% \begin{RTGrid}[...]{number of tasks}{time length}
\begin{RTGrid}[width=10cm]{3}{20}
% \TaskArrDead{task number}{release}{relative deadline}
% \TaskExecution{task number}{run time}{suspend time}

% Task 1
\TaskArrDead{1}{0}{4}
\TaskArrDead{1}{4}{4}
\TaskArrDead{1}{8}{4}

\TaskExecution{1}{0}{1}
\TaskExecution{1}{4}{5}
\TaskExecution{1}{8}{9}

% Task 2
\TaskArrDead{2}{0}{6}
\TaskArrDead{2}{6}{6}

\TaskExecution{2}{1}{3}
\TaskExecution{2}{6}{8}

% Task 3
\TaskArrDead{3}{0}{8}

\TaskExecution{3}{3}{4}
\TaskExecution{3}{5}{6}
\TaskExecution[color=red]{3}{9}{10}

\end{RTGrid}

The 3\textsuperscript{rd} task's deadline was missed on the above chronogram (pointed out by the red execution). \\

\begin{RTGrid}[width=10cm]{3}{22}
% \TaskArrDead{task number}{release}{relative deadline}
% \TaskExecution{task number}{run time}{suspend time}

% Task 1
\TaskArrDead{1}{0}{7}
\TaskArrDead{1}{7}{7}
\TaskArrDead{1}{14}{7}

\TaskExecution{1}{0}{2}
\TaskExecution{1}{7}{9}
\TaskExecution{1}{14}{16}

% Task 2
\TaskArrDead{2}{0}{11}
\TaskArrDead{2}{11}{11}

\TaskExecution{2}{2}{4}
\TaskExecution{2}{11}{13}

% Task 3
\TaskArrDead{3}{0}{13}

\TaskExecution{3}{4}{7}
\TaskExecution{3}{9}{11}
\TaskExecution{3}{13}{14}
\TaskExecution{3}{16}{20}

\end{RTGrid}

There are no missed deadlines visible above in such a short time interval. However, it was proved theoretically that this task set is not schedulable with Rate-Monotonic scheduling.\\

We wrote a \textbf{calibrate()} function which returns the value of one unit of time (in count units). We selected an arbitrary VALUE constant (20000000), which is used as a \emph{for} loop's number of iteration in which \textbf{nop()} instructions are executed. The value returned by this function is then used to compute each task period (\emph{Period[i] * time\_unit}).\\

Each time we want to generate one unit of time to simulate a task's worst case execution time (C), we can use the \textbf{run\_for\_1\_time\_unit()} function which just contains a \emph{for} loop such as the one described above (VALUE iterations, nop() instructions).

\paragraph{Output}
\

\begin{lstlisting}[keywordstyle=\color{black}, caption={Output (reformatted) RM 1\textsuperscript{st} set}]
Start task 0	759 ns
End task 0		12501474 ns
Task length 0	12500714 ns

Start task 1	12503517 ns
End task 1		37502458 ns
Task length 1	24998942 ns

Start task 2	37503485 ns
	Start task 0	50158835 ns
	End task 0		62662460 ns
	Task length 0	12503625 ns

	Start task 1	75238236 ns
	End task 1		100237847 ns
	Task length 1	24999610 ns

	Start task 0	100317636 ns
	End task 0		112817246 ns
	Task length 0	12499610 ns
End task 2		125027074 ns
Task length 2	87523590 ns

...
\end{lstlisting}

\begin{lstlisting}[keywordstyle=\color{black}, caption={Output (reformatted) RM 2\textsuperscript{nd} set}]
Start task 0	847 ns
End task 0		25001304 ns
Task length 0	25000458 ns

Start task 1	25003181 ns
End task 1		50002115 ns
Task length 1	24998935 ns

Start task 2	50003165 ns
	Start task 0	87943696 ns
	End task 0		112943117 ns
	Task length 0	24999420 ns
End task 2		137507069 ns
Task length 2	87503906 ns

Start task 1	138197334 ns
End task 1		163197098 ns
Task length 1	24999765 ns

Start task 2	163324025 ns
	Start task 0	175887354 ns
	End task 0		200886195 ns
	Task length 0	24998842 ns
End task 2		250827616 ns
Task length 2	87503591 ns

Start task 0	263831108 ns
End task 0		288830802 ns
Task length 0	24999695 ns

Start task 1	288832314 ns
End task 1		313831213 ns
Task length 1	24998900 ns

Start task 2	326648002 ns
	Start task 0	351774670 ns
	End task 0		376773728 ns
	Task length 0	24999058 ns
End task 2		414152606 ns
Task length 2	87504603 ns

Start task 1	414591681 ns
End task 1		439591427 ns
Task length 1	24999746 ns

Start task 0	439718388 ns
End task 0		464717746 ns
Task length 0	24999357 ns

Start task 2	489971932 ns
	Start task 0	527661982 ns
	End task 0		552660928 ns
	Task length 0	24998945 ns

	Start task 1	552788734 ns
	End task 1		577787627 ns
	Task length 1	24998892 ns
End task 2		602481868 ns
Task length 2	112509936 ns

...
\end{lstlisting}

\paragraph{Source code}
\

\lstinputlisting{../TP1/task_exo3_rm.c}

\subsubsection{Earliest deadline first scheduling}
% TODO
