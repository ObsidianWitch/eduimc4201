\subsection{Ex3 - Scheduling}

\subsubsection{Schedulability}

To study the schedulability of a task set, the processor load is computed. This value allows us to determine with which algorithm the task set is schedulable. \\

\noindent For the first task set, the processor load is the following : \\

$U_1 = \sum_{i=1}^{n} \frac{C_i}{T_i} = 0.958$ \\

The value of $U_1$ is greater than $n(2^{\frac{1}{n}}-1)$ (with $n$ the number of tasks) which is equal to 0.779. It is then not schedulable with RM, but $U_1$ is still lesser than 1, making the task set schedulable with EDF. \\

When it comes to the second task set, the processor load's value is $U_2 = 0.852$. Like the first task set, it is schedulable with EDF, but not with RM. \\

\begin{center}
	\begin{tabular}{|c|c|c|c|}
		\hline
		\multicolumn{4}{|c|}{\textbf{Set 1}} \\
		\hline
		\, & C & Period & Deadline \\
		\hline
		T1 & 1 & 4 & 4 \\
		\hline
		T2 & 2 & 6 & 6 \\
		\hline
		T3 & 3 & 8 & 8 \\
		\hline
	\end{tabular}
	\quad
	\begin{tabular}{|c|c|c|c|}
		\hline
		\multicolumn{4}{|c|}{\textbf{Set 2}} \\
		\hline
		\, & C & Period & Deadline \\
		\hline
		T1 & 2 & 7 & 7 \\
		\hline
		T2 & 2 & 11 & 11 \\
		\hline
		T3 & 5 & 13 & 13 \\
		\hline
	\end{tabular}
\end{center}

\subsubsection{Rate-monotonic scheduling}
% TODO

% \begin{RTGrid}[...]{number of tasks}{time length}
\begin{RTGrid}[width=10cm]{3}{20}
% \TaskArrDead{task number}{release}{relative deadline}
% \TaskExecution{task number}{run time}{suspend time}

% Task 1
\TaskArrDead{1}{0}{4}
\TaskArrDead{1}{4}{4}
\TaskArrDead{1}{8}{4}

\TaskExecution{1}{0}{1}
\TaskExecution{1}{4}{5}
\TaskExecution{1}{8}{9}

% Task 2
\TaskArrDead{2}{0}{6}
\TaskArrDead{2}{6}{6}

\TaskExecution{2}{1}{3}
\TaskExecution{2}{6}{8}

% Task 3
\TaskArrDead{3}{0}{8}

\TaskExecution{3}{3}{4}
\TaskExecution{3}{5}{6}
\TaskExecution[color=red]{3}{9}{10}

\end{RTGrid}

The 3\textsuperscript{rd} task's deadline was missed on the above chronogram (pointed out by the red execution). \\

\begin{RTGrid}[width=10cm]{3}{22}
% \TaskArrDead{task number}{release}{relative deadline}
% \TaskExecution{task number}{run time}{suspend time}

% Task 1
\TaskArrDead{1}{0}{7}
\TaskArrDead{1}{7}{7}
\TaskArrDead{1}{14}{7}

\TaskExecution{1}{0}{2}
\TaskExecution{1}{7}{9}
\TaskExecution{1}{14}{16}

% Task 2
\TaskArrDead{2}{0}{11}
\TaskArrDead{2}{11}{11}

\TaskExecution{2}{2}{4}
\TaskExecution{2}{11}{13}

% Task 3
\TaskArrDead{3}{0}{13}

\TaskExecution{3}{4}{7}
\TaskExecution{3}{9}{11}
\TaskExecution{3}{13}{14}
\TaskExecution{3}{16}{20}

\end{RTGrid}

There are no missed deadlines visible above in such a short time interval. However, it was proved theoretically that this task set is not schedulable with Rate-Monotonic scheduling.

\subsubsection{Earliest deadline first scheduling}
% TODO
