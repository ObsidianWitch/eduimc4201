\subsection{Ex1 - Semaphores}
The goal of this exercise is to create two tasks which must execute their instructions in a precise order (see listing \ref{lst:exp_out}).

\begin{multicols}{2}
\begin{lstlisting}[numbers=none,caption = {Task 1}]
Print 'T1'
Print 'T1 after T2'
\end{lstlisting}

\begin{lstlisting}[numbers=none, caption={Task 2}]
Print 'T2 starts after T1'
Print 'T2 ends'
\end{lstlisting}
\end{multicols}

\begin{lstlisting}[numbers=none, caption={Expected output}, label={lst:exp_out}]
Print 'T1'
Print 'T2 starts after T1'
Print 'T1 after T2'
Print 'T2 ends'
\end{lstlisting}
\

This can be done with (binary) semaphores which are created in rtai with \textbf{rt\_typed\_sem\_init()} and destroyed with \textbf{rt\_sem\_delete()}. To use a resource (lock a binary semaphore) \textbf{rt\_sem\_wait()} is used, and to give a resource (unlock a binary semaphore) \textbf{rt\_sem\_signal()} is used.\\

To obtain the wanted output, we initialize two semaphores to 0 and use them in the following manner :
\begin{multicols}{2}
\begin{lstlisting}[numbers=none,caption = {Task 1}]
Print 'T1'
signal(sem2)
wait(sem1)
Print 'T1 after T2'
signal(sem2)
\end{lstlisting}

\begin{lstlisting}[numbers=none, caption={Task 2}]
wait(sem2)
Print 'T2 starts after T1'
signal(sem1)
wait(sem2)
Print 'T2 ends'
\end{lstlisting}
\end{multicols}

\subsubsection{Output}
\begin{lstlisting}[keywordstyle=\color{black}]
[task 1] init return code 0 by program /home/rtai/Desktop/rtai_tp/tp2/task.c
[task 2] init return code 0 by program /home/rtai/Desktop/rtai_tp/tp2/task.c
T1
T2 starts after T1
T1 after T2
T2 ends

T1
T2 starts after T1
T1 after T2
T2 ends

T1
T2 starts after T1
T1 after T2
T2 ends

...
\end{lstlisting}

\subsubsection{Source code}
\lstinputlisting{../TP2/task_exo1.c}
