\section{Introduction}
This lab aims to apply the theoretical knowledge acquired in the real time course. We used RTAI, which is a real time kernel patch for Linux.\\

As for the specific tools we used, we discovered Linux CNC \cite{cite:linuxcnc} which is a Debian Wheezy distribution containing a kernel already patched with RTAI 3.9.1. It allowed us to experiment with rtai on our own computers by using it as a live distribution. On a side note, running rtai in a virtual machine may not be such a bad idea and needs further consideration, since virtualization softwares such as VirtualBox reserve one or more CPUs (can be adjusted) for a virtual machine.\\

\begin{framehint}
An archive containing all the source files can be found here :\\    \href{https://bitbucket.org/obside/real_time_tps/get/master.zip}{https://bitbucket.org/obside/real\_time\_tps/get/master.zip}
\end{framehint}
