\section{Introduction}
% TODO

As for the tools we used, we discovered Linux CNC \cite{cite:linuxcnc} which is a Debian Wheezy distribution containing a kernel already patched with rtai 3.9.1. It allowed us to experiment with rtai on our own computers by using it as a live distribution. On a side note, running rtai in a virtual machine may not be such a bad idea and needs further consideration, since virtualization softwares such as VirtualBox reserve one or more CPUs (can be adjusted) for a virtual machine.\\

\begin{framehint}
    An archive containing all the source files can be found here :\\
    \href{TODO}{TODO} % TODO public bitbucket repository -> https://bitbucket.org/obside/real_time_tps/get/TODO.zip
\end{framehint}
